\documentclass[a4paper,12pt]{article}
\usepackage[utf8]{inputenc}
\usepackage[normalem]{ulem}
\usepackage{tikz}
\usepackage{lmodern}
\usepackage{geometry}
\geometry{margin=1in}
\usepackage{seqsplit}

% === Primary key: bold with slightly lower solid underline ===
\newcommand{\bdunderline}[1]{%
  \tikz[baseline=(X.base)]{
    \node[inner sep=0pt,outer sep=0pt](X){\textbf{#1}};
    \draw[line width=0.5pt] 
      ([yshift=-1.5pt]X.south west)--([yshift=-1.5pt]X.south east);
  }%
}
\newcommand{\entity}[1]{\textbf{\fontsize{13}{15}\selectfont #1}}

% === Foreign key: bold with dashed underline (- - -) ===
\newcommand{\bdashuline}[1]{%
  \tikz[baseline=(X.base)]{
    \node[inner sep=0pt,outer sep=0pt](X){\textbf{#1}};
    \draw[dash pattern=on 2pt off 2pt, line width=0.4pt] 
      ([yshift=-1.5pt]X.south west)--([yshift=-1.5pt]X.south east);
  }%
}

% === preserves newlines in parentheses ===
\newenvironment{schema}{
  \par\vspace{4pt}\noindent\ttfamily
  \obeylines % <-- makes LaTeX respect line breaks
}{\par\vspace{6pt}}

% === Both primary + foreign: bold with solid + dashed lines spaced apart ===
\newcommand{\bbothuline}[1]{%
  \tikz[baseline=(X.base)]{
    \node[inner sep=0pt,outer sep=0pt](X){\textbf{#1}};
    % solid underline (primary)
    \draw[line width=0.5pt] 
      ([yshift=-1.5pt]X.south west)--([yshift=-1.5pt]X.south east);
    % dashed underline (foreign)
    \draw[dash pattern=on 2pt off 2pt, line width=0.4pt] 
      ([yshift=-3.5pt]X.south west)--([yshift=-3.5pt]X.south east);
  }%
}

\begin{document}

\section*{\centering \underline{Database Schema}}

% === Employee Entity
\subsection*{}
\entity{Employee} (
\begin{schema}
\bdunderline{employee\_id}, 
first\_name, last\_name, 
national\_id, email,
gender, salary, 
address, official\_day\_off, 
emergency\_phone\_number, emergency\_name,
hire\_date, last\_working\_date, 
years\_of\_experience, type\_of\_contract,
employment\_status, entitlement\_to\_annual\_leave\_balance, 
entitlement\_to\_accidental\_leave\_balance
\end{schema})

% === PhoneNumber Multi-valued Attribute
\subsection*{}
\entity{PhoneNumber}(
\begin{schema} \bbothuline{employee\_id}, \bdunderline{number}
)
\texttt{PhoneNumber.employee\_id} \textit{references} \texttt{Employee.employee\_id}
\end{schema}

% === Role Entity
\subsection*{}
\entity{Role} (
\begin{schema}
\bdunderline{role\_name}, 
title, rank,
description, base\_salary,
initial\_annual\_leave, initial\_accidental\_leave,
salary\_increase\_factor, overtime\_factor 
\end{schema})

% === Department Entity
\subsection*{}
\entity{Department} (
\begin{schema}
\bdunderline{dept\_name}, location
\end{schema})

% === Replaces Relationship
\subsection*{}
\entity{Replaces}(
\begin{schema} 
    employee\_on\_leave, replacement\_employee,
    starting\_date, ending\_date
)
\texttt{Replaces.employee\_on\_leave} \textit{references} \texttt{Employee.employee\_id}
\texttt{Replaces.key replacement\_employee} \textit{references} \texttt{Employee.employee\_id}
\end{schema}

% === works_in Relationship
\subsection*{}
\entity{works\_in}(
\begin{schema} 
\bbothuline{role\_name}, \bbothuline{dept\_name}
)
\texttt{works\_in.role\_name} \textit{references} \texttt{Role.role\_name}
\texttt{works\_in.dept\_name} \textit{references} \texttt{Department.dept\_name}
\end{schema}

% === has_role Relationship
\subsection*{}
\entity{has\_role}(
\begin{schema}
\bbothuline{employee\_id}, \bbothuline{role\_name})
)
\texttt{has\_role.role\_name} \textit{references} \texttt{Role.role\_name}
\texttt{has\_role.employee\_id} \textit{references} \texttt{Employee.employee\_id}
\end{schema}

% === Leave Entity
\subsection*{}
\entity{Leave}(
\begin{schema}
    \bdunderline{request\_id}, 
    \bdashuline{employee\_id}, 
    date\_of\_request, approval\_status, 
    start\_date, end\_date, 
    total\_num\_of\_days
)
\texttt{Leave.employee\_id} \textit{references} \texttt{Employee.employee\_id}
\textit{Where:} \texttt{Leave.total\_num\_of\_days = Leave.end\_date - Leave.start\_date}
\end{schema}

% === Accidental_Leave Entity
\subsection*{}
\entity{Accidental\_Leave}(
\begin{schema}
    \bbothuline{request\_id}
)
\texttt{Accidental\_Leave.request\_id} \textit{references} \texttt{Leave}
\end{schema}

% === Annual_Leave Entity
\subsection*{}
\entity{Annual\_Leave}(
\begin{schema}
    \bbothuline{request\_id}
)
\texttt{Annual\_Leave.request\_id} \textit{references} \texttt{Leave}
\end{schema}

% === Compensation_Leave Entity
\subsection*{}
\entity{Compensation\_Leave}(
\begin{schema}
    \bbothuline{request\_id}, 
    date\_of\_extra\_workday, reason
)
\texttt{Compensation\_Leave.request\_id} \textit{references} \texttt{Leave}
\end{schema}

% === Medical_Leave Entity
\subsection*{}
\entity{Medical\_Leave}(
\begin{schema}
    \bbothuline{request\_id}, 
    insurance\_status, disability\_details
)
\texttt{Medical\_Leave.request\_id} \textit{references} \texttt{Leave}
\end{schema}

% === Sick_Leave Entity
\subsection*{}
\entity{Sick\_Leave}(
\begin{schema}
    \bbothuline{request\_id}
)
\texttt{Sick\_Leave.request\_id} \textit{references} \texttt{Medical\_Leave}
\end{schema}

% === Maternity_Leave Entity
\subsection*{}
\entity{Maternity\_Leave}(
\begin{schema}
    \bbothuline{request\_id}
)
\texttt{Maternity\_Leave.request\_id} \textit{references} \texttt{Medical\_Leave}
\end{schema}

% === Document Entity
\subsection*{}
\entity{Document}(
\begin{schema}
    \bdunderline{document\_id}, 
    \bdashuline{employee\_id}, 
    type, size, 
    creation\_date, expiry\_date, 
    status, storage\_location, 
    description, filename
)
\texttt{Document.employee\_id} \textit{references} \texttt{Employee.employee\_id}
\end{schema}

% === Performance Entity

\subsection*{}
\entity{Performance}(
\begin{schema}
    \bdunderline{performance\_id}, 
    \bdashuline{employee\_id}, 
    rating
)
\texttt{Performance.employee\_id} \textit{references} \texttt{Employee.employee\_id}
\end{schema}

% === Performance_Comments Multi-valued Attribute
\subsection*{}
\entity{Performance\_Comments}(
\begin{schema}
    \bbothuline{performance\_id}, 
    \bdunderline{comment}
)
\texttt{Performance\_Comments.performance\_id} \textit{references} \texttt{Performance.performance\_id}
\end{schema}

% === Payroll Entity
\subsection*{}
\entity{Payroll}(
\begin{schema}
    \bdunderline{payroll\_id}, 
    \bdashuline{employee\_id}, 
    bonuses\_amount, final\_salary\_amount,
    deductions\_amount, payment\_date 
)
\texttt{Payroll.employee\_id} \textit{references} \texttt{Employee.employee\_id}
\end{schema}

% === Payroll_Comments Multi-valued Attribute
\subsection*{}
\entity{Payroll\_Comments}(
\begin{schema}
    \bbothuline{payroll\_id}, 
    \bdunderline{comment}
)
\texttt{Payroll\_Comments.payroll\_id} \textit{references} \texttt{Payroll.payroll\_id}
\end{schema}

% === Attendance Entity
\subsection*{}
\entity{Attendance}(
\begin{schema}
    \bdunderline{attendance\_record\_id}, 
    \bdashuline{employee\_id}, 
    date, check\_in\_time,
    check\_out\_time, total\_duration 
)
\texttt{Attendance.employee\_id} \textit{references} \texttt{Employee.employee\_id}
\textit{Where:} {\small \texttt{Attendance.total\_duration = Attendance.check\_out\_time -Attendance.check\_in\_time}}
\end{schema}

% === Deduction Entity
\subsection*{}
\entity{Deduction}(
\begin{schema}
    \bdunderline{deduction\_id}, 
    \bdashuline{employee\_id}, 
    amount, type,
    date, status 
)
\texttt{Deduction.employee\_id} \textit{references} \texttt{Employee.employee\_id}
\end{schema}

\end{document}
